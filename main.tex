\documentclass[12pt]{article}

%%% Packages %%%

\usepackage{amsmath}
\usepackage{amssymb}
\usepackage{amsthm}

\usepackage[dvipsnames]{xcolor}

%%% Formatting/Configs %%%

\setlength{\parindent}{0pt}
\setlength{\parskip}{6pt}

%%% Macros %%%

\newcommand{\N}{\mathbb{N}}
\newcommand{\Z}{\mathbb{Z}}
\newcommand{\Q}{\mathbb{Q}}
\newcommand{\R}{\mathbb{R}}
\newcommand{\C}{\mathbb{C}}

\newcommand{\X}{\mathfrak{X}}
\newcommand{\Cinf}{C^\infty}
\newcommand{\T}{\mathrm{T}}

\newcommand{\ccdot}[1][]{\,\cdot\,}

\let\d\relax
\DeclareMathOperator{\d}{d}
\newcommand{\di}[1][]{\operatorname{d}_{#1}\def\temp{#1}\!}
\DeclareMathOperator{\id}{id}


%%% Elias' crap %%%

\newcommand{\todo}[1]{\textcolor{NavyBlue}{[todo: #1]}}
\newcommand{\beware}[1]{\textcolor{Red}{[beware: #1]}}
\newcommand{\question}[1]{\textcolor{Green}{[question: #1]}}

%%% Environments %%%

\theoremstyle{definition}
\newtheorem*{definition}{Definition}
\newtheorem*{example}{Example}
\newtheorem*{notation}{Notation}

\theoremstyle{plain}
\newtheorem*{thm}{Theorem}
\newtheorem*{prop}{Proposition}
\newtheorem*{lemma}{Lemma}
\newtheorem*{cor}{Corrolary}

\theoremstyle{remark}
\newtheorem*{remark}{Remark}

\begin{document}

%%% MAX %%%

	\section{Partionifolds}
		A partionifold on a manifold $M$ is a partition of $M$ into connected immersed submanifolds called leaves. \bigskip

		In the context of partionifolds we define the following map:
%
		\begin{align*}
			L_\bullet : M \to \mathcal{P}(M), \quad m \mapsto L_m
		\end{align*}

		where $L_m$ is the leaf containing the point $m$. \bigskip

		You can have curves with a non-vanishing derivative which switch between leaves. Let the partionifold be given by smooth hooks (to pick up spagetthi with, don't worry they won't fall off) and the lines $t \mapsto (t, c)$ for all $c \leq 0$. Then the curve which follows one hook to the edge and then continuous along the $x$-axis is non-vanishing and switches between leaves.

%%% ELIAS %%%

	A smooth curve $\gama: \R \to M$ with a derivative vanishing for some $t$ at some point can then switch between any leaves that have the point $m = \gamma(t)$ in their adherence

	Questions:
	\begin{itemize}
		\item can a curve with a non-vanishing derivative switch between leaves ? Yes
		\item why are we interested in those curves?
		\item what about a derivative vanishing on some interval? Not interesting: the path just stops in this time period, so we only get one possible branching point anyway.
	\end{itemize}

\end{document}
