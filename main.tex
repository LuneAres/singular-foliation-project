\documentclass[12pt]{article}

%%% Packages %%%

\usepackage{amsmath}
\usepackage{amssymb}
\usepackage{amsthm}

\usepackage[dvipsnames]{xcolor}

%%% Formatting/Configs %%%

\setlength{\parindent}{0pt}
\setlength{\parskip}{6pt}

%%% Macros %%%

\newcommand{\N}{\mathbb{N}}
\newcommand{\Z}{\mathbb{Z}}
\newcommand{\Q}{\mathbb{Q}}
\newcommand{\R}{\mathbb{R}}
\newcommand{\C}{\mathbb{C}}

\newcommand{\X}{\mathfrak{X}}
\newcommand{\Cinf}{C^\infty}
\newcommand{\T}{\mathrm{T}}

\newcommand{\ccdot}[1][]{\,\cdot\,}

\let\d\relax
\DeclareMathOperator{\d}{d}
\newcommand{\di}[1][]{\operatorname{d}_{#1}\def\temp{#1}\!}
\DeclareMathOperator{\id}{id}


%%% Elias' crap %%%

\newcommand{\todo}[1]{\textcolor{NavyBlue}{[todo: #1]}}
\newcommand{\beware}[1]{\textcolor{Red}{[beware: #1]}}
\newcommand{\question}[1]{\textcolor{Green}{[question: #1]}}

%%% Environments %%%

\theoremstyle{definition}
\newtheorem*{definition}{Definition}
\newtheorem*{example}{Example}
\newtheorem*{notation}{Notation}

\theoremstyle{plain}
\newtheorem*{thm}{Theorem}
\newtheorem*{prop}{Proposition}
\newtheorem*{lemma}{Lemma}
\newtheorem*{cor}{Corrolary}

\theoremstyle{remark}
\newtheorem*{remark}{Remark}

\begin{document}

%%% MAX %%%

	\section{Partionifolds}
		A partionifold on a manifold $M$ is a partition of $M$ into connected immersed submanifolds called leaves. \bigskip

		In the context of partionifolds we define the following map:
%
		\begin{align*}
			L_\bullet : M \to \mathcal{P}(M), \quad m \mapsto L_m
		\end{align*}

		where $L_m$ is the leaf containing the point $m$. \bigskip

		You can have curves with a non-vanishing derivative which switch between leaves. Let the partionifold be given by smooth hooks (to pick up spagetthi with, don't worry they won't fall off) and the lines $t \mapsto (t, c)$ for all $c \leq 0$. Then the curve which follows one hook to the edge and then continuous along the $x$-axis is non-vanishing and switches between leaves.

\newpage
%%% ELIAS %%%

\section{Notations and definitions}

	In this section we recall some of the basic definitions of differential geometry and introduce the notation we're gonna use in the rest of the report, especially about tangent spaces vector fields.

	\todo{preliminary notations}

	Now we recall one of the definitions of a tangent vector.

	\begin{definition}
		Let $m \in M$.
		We call \textbf{tangent space of $M$ at the point $m$} (written $\T_m{M}$) the set of ``\emph{directional derivatives}'', or rather ``\emph{path derivatives}'' (the direction is described by a smooth path in $M$). \\
		Meaning that any \textbf{tangent vector} $v \in \T_m{M}$ is a function $v: \Cinf(M) \to \R$ defined by
		$$
			\forall f \in \Cinf(M), \quad v(f) = \frac{\d}{\di{t}} f(\gamma(t)) \Big|_{t=0}
		$$
		for a path $\gamma \in \Cinf(\R, M)$ such that $\gamma(0) = m$.
	\end{definition}

	\begin{notation}
		Let $\gamma \in \Cinf(J, M)$ (where $J \subset \R$ is an open subset, often taken to be an interval), and $t_0 \in J$.
		As a shorthand for the path derivative along $\gamma$ at time $t_0$ of a function $f \in \Cinf(M)$ we write 
		$$
			\frac{\partial f}{\partial \gamma}(t_0) := \frac{\d}{\di{t}} f(\gamma(t)) \Big|_{t=t_0}
		$$

		Meaning that for any $m \in M$ a tangent vector $v \in \T_m{M}$ is of the form
		$$
			v = \frac{\partial \bullet}{\partial \gamma}(0)
		$$
		for a path $\gamma \in \Cinf(\R, M)$ such that $\gamma(0) = m$.
	\end{notation}

	\todo{local form, standardization}

	Let's define the tangent bundle quickly

	\todo{tangent bundle}

\newpage
~
\newpage

\section{Random notes}

	A smooth curve $\gamma: \R \to M$ with a derivative vanishing for some $t$ at some point can then switch between any leaves that have the point $m = \gamma(t)$ in their adherence

	Questions:
	\begin{itemize}
		\item can a curve with a non-vanishing derivative switch between leaves ? Yes
		\item why are we interested in those curves?
		\item what about a derivative vanishing on some interval? Not interesting: the path just stops in this time period, so we only get one possible branching point anyway.
		\item what if you don't have a whole partition?
	\end{itemize}


	\subsection{Tangent space and vector fields}

		To (maybe) put in there:
		\begin{itemize}
			\item tangent vector: path derivative, local/coordinate version, (equivalence class of paths)
			\item tangent map: $\phi: M \to N$, $\T{\phi}: \T{M} \to \T{N}$
			\item tangent bundle: charts are $\big( \T(U_\alpha), \T(\phi_\alpha) \big)$ for the atlas $\big\{ (U_\alpha, \phi_\alpha) \big\}_\alpha$ of $M$
			\item vector field: smooth map $X: M \to \T{M}$, local version as directional derivations, action on $\Cinf(M)$ ($X: M \times \Cinf(M) \to \R$, so we can also see it as $X: \Cinf(M) \to \Cinf(M)$ \todo{take a look at the smoothness involved in the manipulation (oh, it probably need $\Cinf$ and not just ``differentiable'' for the image to still be differentiable)})
			\item integral curve
			\item Lie bracket
		\end{itemize}

		For any $m \in M$ the tangent space $\T_m{M}$ is the set of ``directional derivatives'' (the direction is given by a path in $m$).
		Meaning that any $v \in \T_m{M}$ is a function $v: \Cinf(M) \to \R$ defined by
		$$
			\forall f \in \Cinf(M), \quad v(f) = \frac{\d}{\di{t}} f(\gamma(t)) \Big|_{t=0}
		$$
		for a path $\gamma \in \Cinf(\R, M)$ such that $\gamma(0) = m$.

		\todo{A word about equivalence classes of paths?}

		\todo{Let's introduce the notation $\frac{\partial{f}}{\partial{\gamma}}(m) := v(f)$}

		If we fix a coordinate chart $(U, x)$ around $m$ then the direction can be explicitely expressed as a vector in $\R^n$.

		For every $t \in \gamma^{-1}(U) \subset \R$ we have
		$$
			f(\gamma(t)) = \big( (f \circ x^{-1}) \circ (x \circ \gamma) \big)(t)
		$$
		where:
		\begin{itemize}
			\item $(x \circ \gamma) \in \Cinf(\R, \R^n)$
			\item $(f \circ x^{-1}) \in \Cinf(\R^n, \R)$
		\end{itemize}
		By the chain rule we then have
		$$
			v(f) = \underbrace{\di[x(m)] \big( f \circ x^{-1} \big)}_{\R^n \to \R} \bigg[ \underbrace{\frac{\d}{\di{t}} \big( x \circ \gamma \big)(t) \Big|_{t=0}}_{\in \R^n} \bigg]
		$$

		If we write
		$$
			\frac{\d}{\di{t}} \big( x \circ \gamma \big)(t) \Big|_{t=0} = \sum_{i=1}^n a_i e_i \in \R^n
		$$
		for some $(a_i)_{1 \leq i \leq n}$ we have
		$$
			v(f) = \sum_{i=1}^n a_i \di[x(m)](f \circ x^{-1})[e_i]
		$$

		We introduce the notations $\frac{\partial{f}}{\partial{x_i}}: U \to \R$ for directional derivative using the coordinate chart $x$ on $U$. We write
		\begin{align*}
			\frac{\partial{f}}{\partial{x_i}}(m) &= \frac{\partial{(f \circ x^{-1})}}{\partial{e_i}}(x(m)) \\
			                                     &= \di[x(m)](f \circ x^{-1})[e_i] \\
			                                     &= \frac{\d}{\di{t}} \big( f \circ x^{-1} \big)(x(m) + t e_i) \Big|_{t=0} \\
			                                     &= \lim_{t \to 0} \frac{\big( f \circ x^{-1} \big)(x(m) + t e_i) - f(m)}{t}
		\end{align*}
		\todo{clarify/choose notations}

		$$
			\frac{\partial{f}}{\partial{x_i}}(m) \quad \frac{\partial{f}}{\partial{x_i}}\Big|_m \quad \frac{\partial}{\partial{x_i}} f \Big|_m \quad \frac{\partial}{\partial{x_i}} \Big|_m f
		$$

		So we can rewrite:
		$$
			v(f) = \sum_{i=1}^n a_i \frac{\partial f}{\partial{x_i}}(m)
		$$

		We can also write $v$ like this:
		$$
			v = \sum_{i=1}^n a_i \frac{\partial \,\bullet\,}{\partial{x_i}}(m)
		$$

		Let $\phi = (\phi_1, \ldots, \phi_n): x(U) \to \phi(x(U))$ any diffeomorphism.\\
		We can consider the chart $(U, y)$ where $y = \phi \circ x$.
		% So $(y_1, \ldots, y_n) = (\phi_1 \circ x, \ldots, \phi_n \circ x)$.

		Using this chart we have
		\begin{align*}
			v(f) &= \di[y(m)] \big( f \circ y^{-1} \big) \bigg[ \frac{\d}{\di{t}} \big( y \circ \gamma \big)(t) \Big|_{t=0} \bigg] \\
			     &= \di[\phi(x(m))] \big( f \circ x^{-1} \circ \phi^{-1} \big) \bigg[ \frac{\d}{\di{t}} \big( \phi \circ x \circ \gamma \big)(t) \Big|_{t=0} \bigg]
		\end{align*}

		So in order to have $v = \frac{\partial{\bullet}}{\partial{y_1}}$ we want
		$$
			\frac{\d}{\di{t}} \big( \phi \circ x \circ \gamma \big)(t) \Big|_{t=0} = e_1
		$$

		Using the chain rule we have
		\begin{align*}
			\frac{\d}{\di{t}} \big( \phi \circ x \circ \gamma \big)(t) \Big|_{t=0} &= \di[x(m)]{\phi} \Big[ {\textstyle\sum_{i=1}^n a_i e_i} \Big] \\
			                                                                       &= \frac{\d}{\di{t}} \phi\big( x(m) + {\textstyle t \sum_{i=1}^n a_i e_i} \big) \Big|_{t=0}
		\end{align*}



		Let's now consider a vector field $X \in \X(M)$ and $X|_U$ its restriction to $U$.

		For any point $p \in U$, we have a family $(a_i(p))_{1 \leq i \leq n}$ such that
		$$
			X(p) = \sum_{i=1}^n a_i(p) \frac{\partial \,\bullet\,}{\partial{x_i}}(p)
		$$

		The goal will be to find a 

		Vector fields as 

\section{Notations}

	\begin{itemize}
		\item ``smooth'' means differentiable?
		\item ``manifold'' means smooth real manifold
		\item We fix $M$ a $n$-dimensional manifold
		\item We will mainly use two notations for coordinate charts on an open subset $U \subset M$:
			\begin{itemize}
				\item $(U, \phi)$ where $\phi: U \to \phi(U) \subset \R^n$ is a diffeomorphism
				\item $(U, x_1, \ldots, x_n)$, whith functions $x_i: U \to \R$ such that the map $x := (x_1, \ldots, x_n) : U \to \R^n$ is a diffeomorphism into its image
			\end{itemize}
		\item \todo{chart around a point}
		\item $\X(M) = \Gamma(TM)$ is the set of vector fields over $M$
		\item for any manifold $N$, we note $\Cinf(N, M)$ the set of smooth functions $N \to M$ \beware{the notation $\Cinf$ isn't coherent with the fact that ``smooth'' only means ``differentiable''}
		\item We will often talk about smooth paths, using $\Cinf(\R, M)$, or $\Cinf(J, M)$ with $J$ an open subset/interval of $\R$
		\item $\Cinf(M)$ the smooth functions $M \to \R$
		\item path-derivative (i.e. tangent vector):$\frac{\partial{f}}{\partial{\gamma}}(t) \in \T_{\gamma(t)}{M}$
		\item path-derivative for some cooridnate chart $(U,x)$ (i.e. tangent vector): $\frac{\partial{f}}{\partial{x_i}}(m) = \di[x(m)](f \circ x^{-1})[e_i] \in \T_m{M}$
	\end{itemize}

\section{Kinda clean?}

	We want

	\begin{lemma}[Straightening lemma]
		Let $X \in \X(M)$. There 
	\end{lemma}

\section{List of steps}
	
	\begin{itemize}
		\item 
	\end{itemize}

	I proposed the notation
	$$
		X(m)(f) = \frac{\partial{f}}{\partial{\gamma_m}}{\color{Red}(m)} = \frac{\d}{\di{t}} f\big( \gamma_m(t) \big) \bigg|_{\color{Red}t=0} \in \T_m{M}
	$$
	for some smooth path $\gamma_m: J \subset \R \to M$ such that $\gamma_m(0) = m$.

	I find the notation strange in the sense that we impose $\gamma_m(0) = m$ and take the path derivative at $t=0$. So I feel it should rather be one of the two following:
	$$
		\frac{\partial{f}}{\partial{\gamma_m}}{\color{Red}(0)} \quad \text{or even simply} \quad \frac{\partial{f}}{\partial{\gamma_m}}
	$$

	The first one as the advantage of being able to write the following definition:

	\begin{definition}
		Let $X \in \X(M)$.
		We say that $\gamma \in \Cinf(J,M)$ (for some $J \subset \R$) is an \textbf{integral curve} of $X$ if for every $t \in J$ we have
		$$
			X(\gamma(t)) = \frac{\partial \bullet}{\partial \gamma}(t)
		$$

		\todo{What happens when $X$ vanishes at some $\gamma(t)$?}
		
		\todo{existence of an integral curve passing through any point where the vector field doesn't vanish: straightening lemma then pullback of straight lines (well, if the vector field vanishes we have the fixed curve)}

		\todo{unicity of such an integral curve: I don't know}

		\todo{def of a complete vector field}

		\todo{def of a maximal integral curve}

		\todo{def integral curve from a point}
	\end{definition}

	\begin{definition}
		Let $X \in \X(M)$ and $t \in \R$.
		Any point $m \in M$ can be associated with the maximal integral curve $\gamma_m: J_m \to \R$ from $m$ ($J_m$ is an open interval around $0$, and $\gamma_m(0)=m$).\\ For any $t$ we can define $U_t \subset M$ the set of $m \in M$ such that $t \in J_m$ \todo{is $U_t$ open?}.
		This is the domain of definition of the map $\phi^X_t: U_t \to M$ dedfined by
		$$
			\phi^X_t(m) := \gamma_m(t)
		$$

		We call the $\phi^X_t$ the \textbf{flow of $X$ at time $t$}.

		\todo{remark: $\phi^X_0 = \id_U$ (can be defined for every point)}

		\todo{if necessary, change 0 for $t_0$}

		\todo{is the flow a diffeomorphism into its image?}

		\todo{existence of a flow around a point (see takes on that in todos below)}

		\todo{using the straightening lemma we can always find a chart $(U^m,x^m)$ around any $m \in M$ such that $X$ is straightened (i.e. $X|_{U_m} = \frac{\partial}{\partial x^m_1}$), and restricting $U_m$ if needed we can consider that $x(U) \subset \R^n$ is a filled open square. Then any point in $p \in U_m$ admits an integral curve defined on the same interval $J_m$. If you take $U$ contained in a compact subset of $\R^n$ then the existence of an interval $J$ satisfying}

		\todo{if $U$ is contained in a compact set, then the existence of an interval $J$ flow}
	\end{definition} 

	\todo{existence of a }

	\begin{definition}
		$[X,Y](f) = X(Y(f)) - Y(X(f))$

		$[X,Y](m)(f) = X(m)(p \mapsto Y(p)(f)) - Y(m)(p \mapsto X(p)(f))$
	\end{definition}

	\todo{flow following interpretation}

\end{document}
